\documentclass[12pt]{article} % 12-point font

\usepackage[margin=1in]{geometry} % set page to 1-inch margins
\usepackage{bm}
\usepackage{amsmath} % for math
\usepackage{amssymb} % like \Rightarrow
\setlength\parindent{0pt} % Suppresses the indentation of new paragraphs.

% Big display
\newcommand{\ds}{\displaystyle}
% Parenthesis
\newcommand{\norm}[1]{\left\lVert#1\right\rVert}
\newcommand{\p}[1]{\left(#1\right)}
\newcommand{\bk}[1]{\left[#1\right]}
\newcommand{\bc}[1]{\left\{#1\right\}}
\newcommand{\abs}[1]{\left|#1\right|}
% Derivatives
\newcommand{\df}[2]{\frac{d#1}{d#2}}
\newcommand{\ddf}[2]{\frac{d^2#1}{d{#2}^2}}
\newcommand{\pd}[2]{\frac{\partial#1}{\partial#2}}
\newcommand{\pdd}[2]{\frac{\partial^2#1}{\partial{#2}^2}}
% Distributions
\newcommand{\Bernoulli}{\text{Bernoulli}}
\newcommand{\Beta}{\text{Beta}}
\newcommand{\Binomial}{\text{Binomial}}
\newcommand{\Categorical}{\text{Categorical}}
\newcommand{\Dirichlet}{\text{Dirichlet}}
\newcommand{\G}{\text{Gamma}}
\newcommand{\InvGamma}{\text{Inv-Gamma}}
\newcommand{\LogNormal}{\text{LogNormal}}
\newcommand{\Normal}{\text{Normal}}
\newcommand{\MvNormal}{\text{MvNormal}}
\newcommand{\Uniform}{\text{Uniform}}
\newcommand{\SkewT}{\mathcal{ST}}
\newcommand{\TruncatedNormal}{\text{TruncatedNormal}}
% Statistics
\newcommand{\E}{\text{E}}
\newcommand{\Var}{\text{Var}}
\newcommand{\iid}{\overset{iid}{\sim}}
\newcommand{\ind}{\overset{ind}{\sim}}
\newcommand{\Ind}[1]{\mathbf{1}\bc{#1}}
\newcommand{\logistic}{\text{logistic}}
\newcommand{\logit}{\text{logit}}

% Graphics
\usepackage{graphicx}  % for figures
\usepackage{float} % Put figure exactly where I want [H]

% Colors
\usepackage[dvipsnames]{xcolor}
\newcommand{\alert}[1]{\color{red}#1 \color{black}}
\newcommand{\todo}[1]{\color{blue}#1 \color{black}}
\newcommand{\edited}[1]{\color{green}#1 \color{black}}

% This project
\newcommand{\true}{\text{TRUE}}

% Uncomment if using bibliography
% Bibliography
% \usepackage{natbib}
% \bibliographystyle{plainnat}

% Adds settings for hyperlinks. (Mainly for table of contents.)
\usepackage{hyperref}
\hypersetup{
  pdfborder={0 0 0} % removes red box from links
}

% Title Settings
\title{Simulation Study}
\author{Arthur Lui}
\date{\today} % \date{} to set date to empty

% MAIN %
\begin{document}

\maketitle

\section{Simulation Study}
We assessed our model through the following simulation study. We first
generated four data sets (I, II, III, IV) according to our model.
Table~\ref{tab:sim-truth} contains the simulation truth of the model
parameters under the four scenarios. Figure~\ref{fig:sim-truth-density}
contains the histogram of the observed data and the true density of the data
in each scenario. The figures also list the observed proportion of zeros in
the datasets. Note that in scenarios I and II, the densities of the
simulation truths are visibly different, while the true proportions of zeros
are the same in both samples. In scenario III, the densities of the two
samples are the same, but the true probabilities of observing zeros differ.
In scenario IV, the densities and probabilities of observing of zeros are
identical in the simulation truth. In each scenario, $N_i=100000$.
$K^\true=4$ for Scenarios 2,3,4; while $K^\true=3$ for Scenario 1. These
scenarios were created to imitate real data we will next analyze. \\

The following priors were used in this analysis. First, we set $K$ to be a
value ine $\bc{2,3,\dots,9}$. Then $\gamma_i\sim\Beta(1, 1)$,
$\bm\eta_i\sim\Dirichlet_K(1/K)$, $\mu_k\sim\Normal(\bar{\mu}, s_\mu^2)$,
$\tau\sim\G(0.5,\text{rate}=1)$, $\omega_k\mid\tau\sim\InvGamma(2.5, \tau)$,
$\nu_k\sim\LogNormal(3, 0.5)$, $\psi_k\sim\Normal(-1, 0.5)$, where,
respectively, $\bar{\mu}$ and $s_\mu$ are the empirical mean and standard
deviation of the data for which $y_{i,n} > 0$. Posterior inference was done
via Gibbs sampling, with a 20000-iteration burn-in, and the next 8000
samples thinned by every other sample collected to yield 4000 samples for
downstream analysis.

\begin{table}[t!]
  \centering
  \begin{tabular}{|c|cccc|}
    \hline 
    & Scenario I & Scenario II & Scenario III & Scenario IV \\
    \hline 
    $\gamma_C$  & 0.1 & 0.1 & 0.1 & 0.15 \\
    $\gamma_T$  & 0.1 & 0.1 & 0.2 & 0.15 \\
    $\bm\eta_C$ & (0.25,0.75,0) & (0.1,0.1,0.5,0.3) & (0.05,0.05,0.5,0.4) & (0.05,0.05,0.5,0.4) \\
    $\bm\eta_T$ & (0.1,0.1,0.8) & (0.1,0.1,0.8,0) & (0.05,0.05,0.5,0.4) & (0.05,0.05,0.5,0.4) \\
    $\bm\mu$    & (-1.5,3.5,5.1,5) & (-1.5,3.5,1.5,4.3) & (-1.5,3.5,5.1,4.3) & (-1.5,3.5,5.1,4.3) \\
    $\bm\sigma$ & (1.6,1.76,1.76,1.6) & (1.6,1.76,1.76,1.6) & (1.6,1.76,1.76,1.6) & (1.6,1.76,1.76,1.6) \\
    $\bm\nu$    & (12,10,10,15) & (12,10,10,15) & (12,10,10,15) & (12,10,10,15) \\
    $\bm\phi$   & (0,-10,-10,0) & (12,10,10,-11) & (0,-10,-10,-11) & (0,-10,-10,-11) \\
    \hline
  \end{tabular}
  \caption{Simulation truth of model parameters under the four scenarios.}
  \label{tab:sim-truth}
\end{table}

\begin{figure}[t!]
  \centering
  \begin{tabular}{cc}
    (a) Scenario I & (b) Scenario II \\
    \includegraphics[scale=0.5]{results/K=4_skewtmix=true_snum=1/img/data.pdf} &
    \includegraphics[scale=0.5]{results/K=4_skewtmix=true_snum=2/img/data.pdf} \\
    (c) Scenario III & (d) Scenario IV \\
    \includegraphics[scale=0.5]{results/K=4_skewtmix=true_snum=3/img/data.pdf} &
    \includegraphics[scale=0.5]{results/K=4_skewtmix=true_snum=4/img/data.pdf}
  \end{tabular}
  \caption{Histograms of simulated data (from $\tilde G_i$) and simulation truth density.}
  \label{fig:sim-truth-density}
\end{figure}


% DICs
\begin{figure}
  \centering
  \begin{tabular}{cc}
    (a) Scenario 1 & (b) Scenario 2 \\
    \includegraphics[scale=.5]{results/img/dic_snum=1.pdf} &
    \includegraphics[scale=.5]{results/img/dic_snum=2.pdf} \\
    (c) Scenario 3 & (d) Scenario 4 \\
    \includegraphics[scale=.5]{results/img/dic_snum=3.pdf} &
    \includegraphics[scale=.5]{results/img/dic_snum=4.pdf}
  \end{tabular}
  \caption{DICs for simulation scenarios for various $K$.}
  \label{fig:sim-study-dics}
\end{figure}

% G-tilde skewt mix
\begin{figure}
  \centering
  \begin{tabular}{cc}
    (a) Scenario 1, $K=5$, Skew-$t$ mixture & (b) Scenario 2, $K=4$, Skew-$t$ mixture \\
    \includegraphics[scale=.5]{results/K=5_skewtmix=true_snum=1/img/post-density.pdf} &
    \includegraphics[scale=.5]{results/K=4_skewtmix=true_snum=2/img/post-density.pdf} \\
    (c) Scenario 3, $K=6$, Skew-$t$ mixture & (d) Scenario 4, $K=5$, Skew-$t$ mixture \\
    \includegraphics[scale=.5]{results/K=6_skewtmix=true_snum=3/img/post-density.pdf} &
    \includegraphics[scale=.5]{results/K=5_skewtmix=true_snum=4/img/post-density.pdf} \\
  \end{tabular}
  \caption{$\tilde{G}_i$ for best $K$ for skew-$t$ mixture in each scenario.}
  \label{fig:sim-study-gtilde-skewt-mix}
\end{figure}

% G-tilde normal mix
\begin{figure}
  \centering
  \begin{tabular}{cc}
    (a) Scenario 1, $K=6$, Normal mixture & (b) Scenario 2, $K=5$, Normal mixture \\
    \includegraphics[scale=.5]{results/K=6_skewtmix=false_snum=1/img/post-density.pdf} &
    \includegraphics[scale=.5]{results/K=5_skewtmix=false_snum=2/img/post-density.pdf} \\
    (c) Scenario 3, $K=6$, Normal mixture & (d) Scenario 4, $K=6$, Normal mixture \\
    \includegraphics[scale=.5]{results/K=6_skewtmix=false_snum=3/img/post-density.pdf} &
    \includegraphics[scale=.5]{results/K=6_skewtmix=false_snum=4/img/post-density.pdf} \\
  \end{tabular}
  \caption{$\tilde{G}_i$ for best $K$ for Normal mixture in each scenario.}
  \label{fig:sim-study-gtilde-normal-mix}
\end{figure}

Figures~\ref{fig:sim-study-Fi-cdf-skewt-mix}-\ref{fig:sim-study-Fi-cdf-normal-mix}
were created as follows. For each posterior sample of
$(\bm\mu,\bm\sigma,\bm\nu,\bm\phi,\bm\eta_C,\bm\eta_T)$, $L=2000$
samples are drawn from the mixtures $\sum_{k=1}^K \SkewT(\cdot \mid
\mu_k,\sigma_k,\nu_k,\phi_k) \cdot \eta_i$, for $i \in \bc{C,T}$, and then
exponentiated. The resulting empirical cumulative distribution functions (ECDF) of the 
exponentiated samples evaluated over an evenly-spaced grid of 100 values from 0
to the 0.999 quantile are then computed. The ECDFs are averaged (point-wise
at each grid point) over all posterior samples to yield the figures.
% CDFs (F)
\begin{figure}
  \centering
  \begin{tabular}{cc}
    (a) Scenario 1, $K=5$, Skew-$t$ mixture & (b) Scenario 2, $K=4$, Skew-$t$ mixture \\
    \includegraphics[scale=.5]{results/K=5_skewtmix=true_snum=1/img/Fi-postmean.pdf} &
    \includegraphics[scale=.5]{results/K=4_skewtmix=true_snum=2/img/Fi-postmean.pdf} \\
    (c) Scenario 3, $K=6$, Skew-$t$ mixture & (d) Scenario 4, $K=5$, Skew-$t$ mixture \\
    \includegraphics[scale=.5]{results/K=6_skewtmix=true_snum=3/img/Fi-postmean.pdf} &
    \includegraphics[scale=.5]{results/K=5_skewtmix=true_snum=4/img/Fi-postmean.pdf} \\
  \end{tabular}
  \caption{$F_i$ for best $K$ for skew-$t$ mixture in each scenario. Area between
  the two curves are (a) 35.72, (b) 8.247, (c) 4.380, (d) 0.0989.}
  \label{fig:sim-study-Fi-cdf-skewt-mix}
\end{figure}


\begin{figure}
  \centering
  \begin{tabular}{cc}
    (a) Scenario 1, $K=6$, Normal mixture & (b) Scenario 2, $K=5$, Normal mixture \\
    \includegraphics[scale=.5]{results/K=6_skewtmix=false_snum=1/img/Fi-postmean.pdf} &
    \includegraphics[scale=.5]{results/K=5_skewtmix=false_snum=2/img/Fi-postmean.pdf} \\
    (c) Scenario 3, $K=6$, Normal mixture & (d) Scenario 4, $K=6$, Normal mixture \\
    \includegraphics[scale=.5]{results/K=6_skewtmix=false_snum=3/img/Fi-postmean.pdf} &
    \includegraphics[scale=.5]{results/K=6_skewtmix=false_snum=4/img/Fi-postmean.pdf} \\
  \end{tabular}
  \caption{$F_i$ for best $K$ for Normal mixture in each scenario. Area
  between the two curves are (a) 35.55, (b) 8.266, (c) 4.425, (d) 0.1224.}
  \label{fig:sim-study-Fi-cdf-normal-mix}
\end{figure}


% Uncomment if using bibliography:
% \bibliography{bib}
\end{document}